\documentclass{article}

\usepackage{polski}
\usepackage[utf8]{inputenc}


\title{\Huge{
		\indexspace Projekt zespołowy \indexspace Aplikacja internetowa do zarządzania finansami \indexspace \textbf{"Where's my money?!"}}}

\author{Patryk Mroczyński \and Oskar Rutkowski \and Jakub Wiśniewski}

\begin{document}
	
	\pagenumbering{gobble}
	\maketitle
	\newpage
	\tableofcontents
	\newpage
	\pagenumbering{arabic}
	
	\section{Charakterystyka ogólna projektu}
	\paragraph{} Aplikacja ma na celu uproszczenie zarządzania budżetem domowym użytkownika. Umożliwia ona dodawanie wydatków i przychodów wraz z przypisaniem im odpowiednich kategorii. Dodawanie może odbywać się albo jako importowanie pliku w formacie CSV, który jest generowany przez użytkownika na stronie banku, albo poprzez ręczne wpisanie przez użytkownika kwoty i przydzielenia kategorii. Użytkownik może definiować subkonta, do których przypisywane są wydatki. Aplikacja umożliwia generowanie raportów i wykresów dla określonego przedziału czasowego z uwzględnieniem tylko wybranych kategorii czy subkont, które są wybrane przez użytkownika. Użytkownik ma dostęp do historii ostatnio generowanych raportów, a także ma możliwość eksportowania ich do wybranego formatu.
	\section{Architektura systemu}
	\paragraph{} System oparty jest na modelu klient-serwer, gdzie klient (łącznie z interfejsem i logiką po stronie klienta) jest uruchamiany w przeglądarce, a na maszynie serwerowej działa relacyjna baza danych oraz program, który wykonuje operacje na bazie danych i udostępnia API, z którego korzysta aplikacja kliencka.
	\section{Wymagania}
	\paragraph{} W rozdziale opisane są wymagania funkcjonalne oraz niefunkcjonalne z podziałem na aktorów.
	\subsection{Wymagania funkcjonalne}
	\begin{enumerate}
		\item Niezalogowany użytkownik może się zarejestrować oraz zalogować.
		\item Zalogowany użytkownik może się wylogować oraz zarządzać ustawieniami konta.
		\item Zalogowany użytkownik może tworzyć subkonta, do których przypisuje wybrane wydatki oraz przychody.
		\item Zalogowany użytkownik może dodawać wyciąg z konta dowolnego banku jako plik w formacie CSV.
		\item Zalogowany użytkownik może ręcznie dodać wpis, wpisując kwotę oraz dobierając kategorię.
		\item Zalogowany użytkownik ma możliwość modyfikacji i usuwania wpisów w aplikacji.
		\item Zalogowany użytkownik może generować raporty z określonego przedziału czasowego dla określonych kategorii i subkont.
		\item Zalogowany użytkownik może eksportować swoje raporty finansowe jako pliki w formacie CSV lub PDF.
		\item Zalogowany użytkownik przy dodawaniu wyciągu konta może wybrać układ pliku CSV między predefiniowanymi bankami lub zdefiniowanym przez siebie układem.
		
	\end{enumerate}
	\subsection{Wymagania niefunkcjonalne}
	\begin{enumerate}
		\item Aplikacja serwerowa napisana w języku Python3.
		\item Aplikacja serwerowa wykorzystuje relacyjną bazę danych PostgreSQL.
		\item Aplikacja nie wymaga od użytkownika instalowania dodatkowego oprogramowania poza aktualną przeglądarką.
		\item Aby użytkownik mógł korzystać z aplikacji musi się poprawanie zalogować na założone w serwisie konto.
		\item Dane użytkownika są przechowywane w bazie danych.
		\item Hasło użytkownika przechowywane jest jako funkcja skrótu SHA-256.
		\item Aplikacja kliencka jest zaprojektowana zgodnie z techniką Responsive Web Design.
		\item API stworzone jest zgodnie z architekturą REST, w formie mikroserwisów.
		\item Raporty mają formę tabeli lub wykresu.
	\end{enumerate}
	\section{Narzędzia, środowiska, biblioteki}
	\paragraph{}Serwer działa w oparciu o system operacyjny Ubuntu Server 16.04 LTS.
	Do zarządzania danymi wykorzystywana jest relacyjna baza danych PostgreSQL w wersji 9.6.8.
	Mikroserwisy działające jako aplikacja serwerowa napisane są w języku Python 3.6.4 wraz z frameworkiem CherryPy 14.0 oraz adapterem Psycopg 2.7.4.
	Aplikacja kliencka to interaktywna strona internetowa, napisana przy pomocy frameworka React dla języka JavaScript, w którym wykorzystywany jest także język HTML5 oraz CSS3 z preprocesorem LESS. Do tworzenia aplikacji jest używany język EcmaScript 2017 w standardzie ECMA-262, który kompilowany jest przy pomocy kompilatora Babel do języka JavaScript. Strona może być poprawnie wyświetlona na dowolnej aktualnej przeglądarce internetowej.
	Narzędzia wykorzystane przy tworzeniu aplikacji to Visual Studio Code 1.20.1, DBeaver Community Edition 5.0.
\end{document}
